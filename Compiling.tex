\documentclass{PionpillNote-art}

\import{sections}{style.tex}

\title{编译原理笔记}
\author{
    Pionpill \footnote{笔名:北岸,电子邮件:673486387@qq.com,Github:\url{https://github.com/Pionpill}} \quad Nekasu \footnote{笔名:小十六的伏特加,电子邮箱:1428147954@qq.com,Github:\url{https://github.com/Nekasu}} \\
    本文档为作者学习《编译原理》\footnote{《程序设计语言 编译原理》:陈火旺,国防工业出版社,2020年印刷}一书时的笔记,\\
}

\date{\today}

\begin{document}

\maketitle

\noindent\textbf{前言:}

本篇笔记用于应付学校期末考试,本人对这方面也没有进行深入研究。内容浅显,不适合考研等深入学习。

编译原理这门课挺难的,本文只能提供一个基于原教程的精炼与复习,针对晦涩的部分或者跳过的理解性例子,会在对应页下给出注释,但有些东西笔者也不是很理解,望谅解。

本文使用 \LaTeX 语言书写,原文编译环境为 TeXLive + VSCode。文中大部分数字编号,超链接,目录均可实现点击跳转。若无法实现基础的目录点击跳转,那可能是你的 pdf 阅读器有问题,建议使用免费的 Adobe Acrobat Reader DC\footnote{下载连接(pro 为付费版本):\url{https://get.adobe.com/cn/reader}}或 Chrome\footnote{以及使用 chrome 核心的 edge 浏览器,小心国产浏览器} 浏览器。得益于 \LaTeX 的强大排版功能以及 TikZ 绘图库,本文所有图片文字均为矢量形式,如有不清楚地方放大即可,如果放大了还不清楚,那你可能拿到的是n手资源。

此外,本篇笔记是对原书的提炼总结,只能辅助原书进行学习,有大量例子等理解性文字并未进行说明,如有需要请购买原书。本笔记文案多为原书摘抄或个人总结,图片为本人使用 TikZ 绘制,若需进行引用,可前往下载 \LaTeX 源代码\footnote{\url{https://github.com/Pionpill/Notebook/tree/Pionpill/Lessons}},并遵守 GPL-v3 协议。

本篇笔记换过一次仓库,之前的版本请前往:\url{https://github.com/Pionpill/Notebook/tree/Pionpill/Lessons}

针对学校考试,\textcolor{imp}{红色}为重要(必考),\textcolor{mark}{蓝色}为次重点,\textcolor{tip}{绿色}为理解性内容,\textcolor{grey}{灰色}为上课跳过或者不重要的内容。

由于原书错误太多,而且内容过于形式化,第四章开始,本文主要参考哈工大编译原理课程\url{https://www.bilibili.com/video/BV1zW411t7YE}。

\date{\today}

\newpage

\tableofcontents
\thispagestyle{empty}
\newpage
\setcounter{page}{1}

\import{sections}{section-1.tex}
\import{sections}{section-2.tex}
\import{sections}{section-3.tex}
\import{sections}{section-4.tex}
\import{sections}{appendix.tex}




\end{document}

