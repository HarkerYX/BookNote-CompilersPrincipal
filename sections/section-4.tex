\section{语法分析——自上而下}
\subsection{语法分析器的功能}

语法分析是编译过程的核心部分。它的任务是在词法分析识别出单词符号串的基础上,分析并判定程序的语法结构是否符合语法规则。

\begin{figure}[H]
    \centering
    \begin{tikzpicture}[thick,font=\small,inner xsep=1em,inner ysep=1ex]
        \node[draw] (1) at (0,0) {词法分析器}; 
        \node[draw] (2) at (5,0) {语法分析器}; 
        \node[draw,text width=4em] (3) at (10,0) {编译程序后续部分}; 
        \node[draw] (4) at (5,-2) {符号表}; 
        \begin{scope}[-Stealth]
            \draw (-3,0) -> (1) node[pos=0.5,above] {源程序};
            \draw (1.east) ++ (0,0.1) -> ++ (2.3,0) node[pos=0.5,above] {单词符号};
            \draw (2.west) ++ (0,-0.1) -> ++ (-2.3,0) node [pos=0.5,below,font=\footnotesize] {取下一单词符号};
            \draw[dashed] (2) -- (3) node [pos=0.5,above] {语法分析树};
            \draw[dashed] (3) -- ++(3,0);
            \draw[Stealth-Stealth] (1.south) -- (4.west);
            \draw[Stealth-Stealth] (2.south) -- (4.north);
            \draw[Stealth-Stealth] (3.south) -- (4.east);
        \end{scope}
    \end{tikzpicture}
    \caption{语法分析器在编译程序中的地位}
    \label{fig:语法分析器在编译程序中的地位}
\end{figure}

语法分析器的工作本质上就是按文法的产生式,识别输入符号串是否为一个句子。这里的输入串指单词符号(文法的终结符)组成的有限序列。

对一个文法,给出一串(终结)符号时,如何判断它是不是该文法的一个句子?此时要判断能否从文法的开始符号出发推导出这个输入串。或者从概念上,就是要建立一棵与输入串相匹配的语法分析树。

按照语法分析树的建立方法,可以将语法分析办法分成两类,即自上而下分析发法和自下而上分析法。

\noindent\textbf{最左推导\footnote{这部分书上没有仔细讲,这里简单说一下,仅供参考。}}

最左推导,即再推导过程中,总是选择每个句型的最左非终结符进行替换。与之对应的逆过程称为最右归约。相对的也有最右推导和最左归约。

在自顶向下分析中,采用的是最左推导与最右归约。




\subsection{自上而下分析面临的问题}

自上而下语法分析,顾名思义,就是从文法的开始符号出发,向下推导,推出句子。自上而下推导的主旨是:对任何输入串,试图用一切可能的办法,从文法开始符号(根结)出发,自上而下地为输入串建立一棵语法树。或者说,为输入串寻找一个最左推导。

下面举一个简单的例子\footnote{参考书 P67-68,这里为本人理解,叙述为白话,更准确但难懂的形式化语言建议参考原书},假定有文法:
\begin{enumerate}
    \item[(1)] S $\rightarrow$ xAy
    \item[(2)] A $\rightarrow$ **|*
\end{enumerate}

以及输入串$\alpha$: x*y,下面构造 $\alpha$ 的语法树。 

\begin{enumerate}
    \item 首先按文法开始符号产生根结 P,并让指示器 IP\footnote{IP 逐个指向输入串 $\alpha$ 中的符号} 指向输入串的第一个符号 x。然后用 S 的规则(此处只有一条)绘制树\footnote{圆圈代表当前匹配的结点}。
    \begin{figure}[H]
        \centering
        \begin{tikzpicture}[font=\small]
            \node at (3.5,-0.75) {IP: x};
            \node (S) at (0,0) {S};
            \node[draw,circle,red!70,text=black] (x) at (-2,-1.5) {x};
            \node (A) at (0,-1.5) {A};
            \node (y) at (2,-1.5) {y};
            \draw[thick] (S) -- (x);
            \draw[thick] (S) -- (A);
            \draw[thick] (S) -- (y);
        \end{tikzpicture}
    \end{figure}

    \item 其次,需要用 S 的子结从左至右匹配整个输入串 $\alpha$,左边第一个子结为终结符 x 和 IP 指向的符号匹配,于是让 IP 调整为指向下一输入符号 *,让第二个字结 A 进行匹配。A 有两个候选,这里试着用第一个候选进行匹配,于是有新的语法树:
    
    \begin{figure}[H]
        \centering
        \begin{tikzpicture}[font=\small]
            \node at (3.5,-1.5) {IP: *};
            \node (S) at (0,0) {S};
            \node (x) at (-2,-1.5) {x};
            \node (A) at (0,-1.5) {A};
            \node (y) at (2,-1.5) {y};
            \node (1)[draw,circle,red!70,text=black] at (-1,-3) {*};
            \node (2) at (1,-3) {*};
            \draw[thick] (S) -- (x);
            \draw[thick] (S) -- (A);
            \draw[thick] (S) -- (y);
            \draw[thick] (A) -- (1);
            \draw[thick] (A) -- (2);
        \end{tikzpicture}
    \end{figure}

    \item 然后,子树 A 的最左子结和 IP 所指的符号 * 相符,然后我们再把 IP 指向下一符号并让 A 的第二个子结进入工作。发现第二子结 * 和 IP 指向的 y 不匹配。因此 A 宣告失败,回溯查看 A 是否有其他候选表达式。注销之前的语法树,绘制新的语法树,IP 恢复为进入 A 时的原值,重新指向 *。
    \begin{figure}[H]
        \centering
        \begin{tikzpicture}[font=\small]
            \node at (3.5,-1.5) {IP: *};
            \node (S) at (0,0) {S};
            \node (x) at (-2,-1.5) {x};
            \node (A) at (0,-1.5) {A};
            \node (y) at (2,-1.5) {y};
            \node (1)[draw,circle,red!70,text=black] at (0,-3) {*};
            \draw[thick] (S) -- (x);
            \draw[thick] (S) -- (A);
            \draw[thick] (S) -- (y);
            \draw[thick] (A) -- (1);
        \end{tikzpicture}
    \end{figure}

    \item 此时将 IP 所指的 * 与 A 的字节 * 匹配,成功匹配后,IP 指向 y,获取语法树第三个字节 y 进行匹配,成功。至此语法树匹配成功,证明是 $\alpha$ 是一个句子。
    
    \begin{figure}[H]
        \centering
        \begin{tikzpicture}[font=\small]
            \node at (3.5,-1.5) {IP: y};
            \node (S) at (0,0) {S};
            \node (x) at (-2,-1.5) {x};
            \node (A) at (0,-1.5) {A};
            \node (y)[draw,circle,red!70,text=black] at (2,-1.5) {y};
            \node (1) at (0,-3) {*};
            \draw[thick] (S) -- (x);
            \draw[thick] (S) -- (A);
            \draw[thick] (S) -- (y);
            \draw[thick] (A) -- (1);
        \end{tikzpicture}
    \end{figure}
\end{enumerate}

这种自上而下分析法存在许多困难和缺点。
\begin{itemize}
    \item \textit{文法左递归} \\
    左递归:存在非终结符 P,有如下表达式:
    \[P \stackrel{+}{\Rightarrow} P \alpha \]
    含有左递归的文法将使上述的分析过程陷入无限循环。因此,使用自上而下分析法必须消除文法的左递归。
    \item \textit{回溯} \\
    如果走了一大段错路,就必须抛弃所有的工作(中间代码,表格记录)重来,浪费时间。因此,最好应该设法消除回溯。
    \item \textit{虚假匹配} \\
    当某一个非终结符用某一候选项匹配成功时,这种成功可能仅是暂时的。这需要更复杂的回溯技术处理这种虚假匹配线性。一般来说,难以消除虚假匹配线性,但从最长的候选项开始匹配,会产生较少的虚假现象。
    \item \textit{出错位置} \\
    当最终报告分析不成功时,难以知道输入串中出错的具体位置。
    \item \textit{穷举的代价} \\
    可以看出,自上而下分析是一种穷举算法。这必然会导致效率低,代价高的问题。甚至严重的低效问题在实践中没有价值。
\end{itemize}

\subsection{LL(1)分析法}

为了克服自上而下分析的缺陷:左递归,回溯。这节将研究递归子程序分析算法及其变种—预测分析法。

\subsubsection{左递归的消除}

假定关于非终结符 P 的规则为\footnote{这里有很好的解释:\url{https://www.bilibili.com/video/BV1zW411t7YE?p=20}}:
\[ P \rightarrow P\alpha|\beta \]
其中,$\beta$ 不以 P 开头,那么,我们可以把 P 的规则改写为如下的非直接左递归形式:
\begin{equation}
    \begin{aligned}
        &P \rightarrow \beta P^{'} \\
        &P^{'} \rightarrow \alpha P^{'} | \epsilon \qquad (\epsilon \text{为空字})  \nonumber
    \end{aligned}
\end{equation}
这种形式和原来的形式是等价的,也即从 P 推出的符号串是相同的。

举个例子:
\begin{equation}
    \begin{aligned}
        E \rightarrow E+T|T \longleftrightarrow 
        \left\{
            \begin{aligned}
                &E \rightarrow TE^{'} \\
                &E^{'} \rightarrow +TE^{'}|\epsilon 
            \end{aligned} 
        \right. \\
        T \rightarrow T*F|F \longleftrightarrow
        \left\{
            \begin{aligned}
                &T \rightarrow FT^{'} \\
                &T^{'} \rightarrow *FT^{'}|\epsilon 
            \end{aligned} 
        \right. \nonumber
    \end{aligned}
\end{equation}

一般而言,假定 P 关于的全部产生式:
\[ P \rightarrow P\alpha_1 | P\alpha_2 | \cdots | P\alpha_m | \beta_1 | \beta_2 | \cdots | \beta_m \]
其中,每个 $\alpha$ 不为 $\epsilon$,每个 $\beta$ 不以 P 开头,那么消除左递归后的规则为:
\begin{equation}
    \begin{aligned}
        &P \rightarrow \beta_1P^{'} | \beta_2P^{'} | \cdots | \beta_nP^{'} \\
        &P^{'} \rightarrow \alpha_1P^{'} | \alpha_2P^{'} | \cdots | \alpha_mP^{'} | \epsilon \nonumber
    \end{aligned}
\end{equation}

使用这个办法,可以将见诸于表面上的直接左递归消除(即把直接左递归改成直接右递归)。但这并不能消除整个文法的左递归性。

如果一个文法不含回路(形如 $P \stackrel{+}{\Rightarrow} P 的推导$ ),也不含以 $\epsilon$ 为右部的产生式,那么,下列算法将保证删除左递归。

\begin{enumerate}
    \item 把文法 G 的所有非终结符按任一种顺序排列成 $P_1,P_2,\cdots,P_n$;按此顺序执行。
    \item 把形如 $P_i \rightarrow P_j \gamma$ 的规则改写成 $P_i \rightarrow \delta_1\gamma | \delta_2\gamma | \cdots | \delta_k\gamma$。\\
    其中 $P_j \rightarrow \delta_1 | \delta_2 | \cdots | \delta_k$ 是关于 $P_j$ 的所有规则。消除关于 $P_i$ 规则的直接左递归性。
    \item 化简上一步得到的文法。即去除那些从开始符号触发永远无法到达的非终结符的产生规则。
\end{enumerate}

\subsubsection{消除回溯,提左因子}

为了消除回溯就必须保证候选项获得成功的匹配。假设任务 A 有 n 个候选项 $\alpha_1,\alpha_2,\cdots,\alpha_n$。A 所面临的第一个输入符号为 a,如果 A 能够根据不同的输入符号指派相应的候选 $\alpha_i$去匹配,就没有回溯了。

令 G 是一个不含左递归的文法,对 G 的所有非终结符的每个候选 $\alpha$ 定义它的终结首符集 FIRST($\alpha$) 为:
\[\text{FIRST}(\alpha) = \{ a | \alpha \stackrel{*}{\Rightarrow} a \cdots, a \in V_{T} \}\]
特别的,若 $\alpha \stackrel{*}{\Rightarrow} \epsilon$,则规定 $\epsilon \in \text{FIRST}(\alpha)$。换句话说,$\text{FIRST}(\alpha)$是 $\alpha$ 的所有可能推导的开头终结符或 $\epsilon$。

如果非终结符 A 的所有候选首符集两两不相交,那么,当要求 A 匹配输入串时,A就能根据它所面临的第一个输入符号 a,准确地指派某一个候选前去执行任务。这个候选就是那个终结首符集含 a 的 $\alpha$。

并非所有enfant的候选的终结首符集都两两不相交。其解决方案是:提取公共左因子。例如,假定关于 A 的规则是:
\[ A \rightarrow \delta \beta_1 | \delta \beta_2 | \cdots | \delta \beta_n | \gamma_1 | \gamma_2 | \cdots | \gamma_m \quad (\text{每个} \gamma \text{不以} \delta \text{开头}) \]
那么,可以把这些规则改写成:
\begin{equation}
    \begin{aligned}
        &A \rightarrow \delta A^{'} | \gamma_1 | \gamma_2 | \cdots | \gamma_m \\
        &A^{'} \rightarrow \beta_1 | \beta_2 | \cdots | \beta_n \nonumber
    \end{aligned}
\end{equation}
经过反复提取左因子,就能够把每个非终结符的所有候选首符集变成两两不相交。

\subsubsection{LL(1)分析条件}

假定 S 是文法 G 的开始符号,对于 G 的任何非终结符 A,我们定义
\[ \text{FOLLOW}(A) = \{ a|S\stackrel{*}{\Rightarrow} \cdots Aa \cdots, a\in V_T \} \]
特别是,若 $S\stackrel{*}{\Rightarrow} \cdots A$,则规定 $\# \in \text{FOLLOW}(A)$。换句话说,$\text{FOLLOW}(A)$ 是所有句型中出现在紧接 A 之后的终结符或 \#。

因此,当非终结符 A 面临输入符号 a,且 a 不属于 A 的任意候选首符集但 A 的某个候选首符集包含 $\epsilon$ 时,只有当 $a \in \text{FOLLOW}(A)$,才可能允许 A 自动匹配。

通过上面一系列讨论,我们可以找出满足构造不带回溯的自上而下分析的文法条件。
\begin{enumerate}
    \item 文法不含左递归。
    \item 文法中每一个非终结符 A 的各个产生式的候选首符集两两不相交。
    \item 对文法中的每个非终结符A,若它存在某个候选首符集包含 $\epsilon$,则
    \[ \text{FIRST(A)} \cap \text{FOLLOW(A)} = \phi \]
\end{enumerate}
如果一个文法 G 满足以上条件,则称该文法 G 为 LL(1) 文法。LL(1) 中第一个 L 表示从左到右扫描输入串,第二个 L 表示最左推导,1 表示分析时每一步只需向前查看一个符号。 

对于一个 LL(1) 文法,可以对其输入串进行有效的无回溯的自上而下分析。假设要用非终结符 A 进行匹配,面临的输入符号为 a,A 的所有产生式为
\[ A \rightarrow \alpha_1 | \alpha_2 | \cdots | \alpha_n \]
\begin{itemize}
    \item 若 $a\in \text{FIRST}(\alpha_1)$,则指派 $\alpha_i$ 去执行匹配任务。
    \item 若 a 不属于任何一个候选首符集,则:
    \begin{itemize}
        \item 若 $\epsilon$ 属于某个 $\text{FIRST}(\alpha_i)$,且 $a\in \text{FOLLOW(A)}$,则让 A 与 $\epsilon$ 自动匹配;
        \item 否则,a的出现是一种语法错误。
    \end{itemize}
\end{itemize}

\noindent\textbf{FIRST 集的计算方法}

FIRST(X) 的定义:
\begin{itemize}
    \item 可以从 X 推导出的所有串首终结符\footnote{串首第一个符号,并且是终结符。}构成的集合。
    \item 如果 X $\stackrel{*}{\Rightarrow} \epsilon$,那么 $\epsilon \in$ FIRST(X)。 
\end{itemize}

FIRST(X) 算法:不断应用下列规则,直到没有新的终结符或$\epsilon$可以被加入到任何 FIRST 集合中为止。
\begin{itemize}
    \item 如果 X 是一个终结符,那么 FIRST(X) = X。
    \item 如果 X 是一个非终结符,有 X $\rightarrow Y_1 \cdots Y_k \in P (k\geq 1) $,那么如果对于某个 $i$,a 在 FIRST($Y_i$) 中,且 $Y_i$ 前的所有非终结符的 FIRST 集中都存在 $\epsilon$。就把 a 加入到 FIRST(X) 中。如果所有 $Y_i$ 的 FIRST 集中都有 $\epsilon$,就把 $\epsilon$ 加入到 FIRST(X)。
    \item 如果 X$\rightarrow \epsilon$,就把 $\epsilon$ 加入到 FIRST(X)。
\end{itemize}

\newpage